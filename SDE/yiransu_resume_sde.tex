%-------------------------
% Resume in Latex
% Author : Sourabh Bajaj
% License : MIT
%------------------------

\documentclass[letterpaper,11pt]{article}

\usepackage{amssymb}
\usepackage{latexsym}
\usepackage[empty]{fullpage}
\usepackage{titlesec}
\usepackage{marvosym}
\usepackage[usenames,dvipsnames]{color}
\usepackage{verbatim}
\usepackage{enumitem}
\usepackage{makecell}
\usepackage[hidelinks]{hyperref}
\usepackage{fancyhdr}
\usepackage[english]{babel}

\pagestyle{fancy}
\fancyhf{} % clear all header and footer fields
\fancyfoot{}
\renewcommand{\headrulewidth}{0pt}
\renewcommand{\footrulewidth}{0pt}

% Adjust margins
\addtolength{\oddsidemargin}{-0.5in}
\addtolength{\evensidemargin}{-0.5in}
\addtolength{\textwidth}{1in}
\addtolength{\topmargin}{-.75in}
\addtolength{\textheight}{1.5in}

\urlstyle{same}

\raggedbottom
\raggedright
\setlength{\tabcolsep}{0in}

% Sections formatting
\titleformat{\section}{
  \vspace{-4pt}\scshape\raggedright\large
}{}{0em}{}[\color{black}\titlerule \vspace{-5pt}]

%-------------------------
% Custom commands
\newcommand{\resumeItem}[1]{
  \item\small{
    {#1}
  }
}

\newcommand{\resumeSubheading}[4]{
  \vspace{-1pt}\item
    \begin{tabular*}{0.97\textwidth}[t]{l@{\extracolsep{\fill}}r}
      \textbf{#1} & #2 \\
      \textit{\small#3} & \textit{\small #4} \\
    \end{tabular*}\vspace{-5pt}
}

\newcommand{\courseworkSubheading}[2]{
	\vspace{-1pt}\item
	\begin{tabular*}{0.97\textwidth}[t]{l@{\extracolsep{\fill}}r}
		\textbf{#1} & #2 \\
	\end{tabular*}\vspace{-5pt}
}

\newcommand{\internSubheading}[2]{
	\vspace{-1pt}\item
	\begin{tabular*}{0.97\textwidth}[t]{l@{\extracolsep{\fill}}r}
		\textbf{#1} & #2 \\
	\end{tabular*}\vspace{-5pt}
}

\newcommand{\resumeSubItem}[2]{\resumeItem{#1}{#2}\vspace{-4pt}}

\renewcommand{\labelitemii}{$\circ$}

\newcommand{\resumeSubHeadingListStart}{\begin{itemize}[leftmargin=*]}
\newcommand{\resumeSubHeadingListEnd}{\end{itemize}}
\newcommand{\resumeItemListStart}{\begin{itemize}}
\newcommand{\resumeItemListEnd}{\end{itemize}\vspace{-5pt}}
\newcommand{\pubItemListStart}{\begin{itemize}\setlength{\itemsep}{0pt} \setlength{\parsep}{0pt} \setlength{\parskip}{0pt}}
\newcommand{\pubItemListEnd}{\end{itemize}}

%-------------------------------------------
%%%%%%  CV STARTS HERE  %%%%%%%%%%%%%%%%%%%%%%%%%%%%


\begin{document}

%----------HEADING-----------------
\begin{tabular*}{\textwidth}{l@{\extracolsep{\fill}}r}
  \textbf{{}{\Large Yiran Su}} & 
  {6805 Wood Hollow Dr}\\
  {\href{mailto:yiransucdr@gmail.com}{$\lozenge$ yiransucdr@gmail.com}
  \href{https://www.linkedin.com/in/su-yiran-a2a146129/}{$\lozenge$ LinkedIn account} $\lozenge$ (512)-999-5939} & {Austin, TX} 
\end{tabular*}


%-----------EDUCATION-----------------
\section{Education}
	\resumeSubHeadingListStart
		\resumeSubheading
			{University of Texas at Austin}{Austin, TX, USA}
			{\makecell[tl]{\textbf{M.S.} in Engineering, \textbf{Software Engineering and System} track, ECE Dept. \textbf{GPA}: 3.73/4 }}
			{Aug. 2019 - May. 2021}
		\resumeSubheading
			{Sun Yat-sen University, School of Data and Computer Science}{Guangzhou, China}
			{\makecell[tl]{\textbf{B.E.} in Network Engineering~~~ 
			\textbf{Overall GPA}: 3.85/5.00, \textbf{Junior GPA}: 4.25/5.00 }}
			{Aug. 2015 - Jun. 2019}
			\resumeItemListStart
			\resumeItem 
				{\textbf{Course Highlights} Data Structure \& Algorithms, Operating System, Computer Network, Web Programming}
			\resumeItemListEnd
\resumeSubHeadingListEnd

%-----------SKILLS-----------------
\section{Skills}
\textbf{Programming Language~}{C++, Python, Java, HTML, CSS, JavaScript, Kotlin, Shell, SQL, GoLang}\\
\textbf{Framework and Tools~}{React Native, Flask, PyTorch, Tensorflow, Docker, MongoDB, Kubernetes}

%-----------AWARDS AND HONORS-----------------
%\section{Awards And Honors}
%\resumeSubHeadingListStart
%\resumeSubItem{Innovative Design Award in the 2018 International Aerial Robotics Competition (2018 IARC)}{}
%\resumeSubItem{Honorable Award in the 2018 Mathematical Contest in Modeling (2018 MCM)}{}
%\resumeSubItem{The First Prize Scholarship of Sun Yat-sen University(Top 5\%)}{}
%\resumeSubHeadingListEnd

%-----------WORK EXPERIENCE-----------------
\section{Intern Experience}
  \resumeSubHeadingListStart
  	\internSubheading
  	{Coherent Logix Inc.}{May. 2020 - Aug. 2020, Austin, USA}
  	\resumeItemListStart
  	\resumeItem
  	{Explored \textbf{nerual network quantization} topics which is able to simplify the original model while keep the original model precision and leverage RAM \& power \& time consumption.}
  	\resumeItem
  	{Applied \textbf{16/8/4-bit} Quantization Aware Training (\textbf{QAT}) for \textbf{ResNet} and \textbf{MobileNet} with \textbf{Tensorflow 2} and controlled model accuracy drop within \textbf{3\%} for 16/8-bit quantization circumstances.}
  	\resumeItem
  	{Developed a set of \textbf{QAT generation \& validation APIs} which enables transform from a non-QATed model to a QATed model and \textbf{compatible} with all sequential/functional models(self-defined, loaded, Keras-imported, etc.).}
  	\resumeItemListEnd
  	
    \internSubheading
      {Tencent Inc.}{Sept. 2018 - Mar. 2019, Shenzhen, China}
      \resumeItemListStart
      	\resumeItem
      	  {Developed a \textbf{pattern-based natural language parsing framework} in \textbf{Python} for a task-oriented Arena of Valor \textbf{chatbot} "Lu Bu (Lv, Bu)", while \textbf{reducing} the average latency by \textbf{27\%} to \textbf{less than 90 ms}.}
      	\resumeItem
          {Deployed the above framework on a \textbf{Tornado} Server, which handled more than \textbf{100k related requests} per day. }
        \resumeItem
          {Designed an \textbf{automatic} user log analyzer (\textbf{Python}) for the chatbot which is able to evaluate high-frequency request, customer stickiness and new feature performance.}
      \resumeItemListEnd
      \resumeSubheading
      {Graduate Teaching Assistant}{Austin, USA}
      {EE 422C Software Design and Implementation (\textbf{Java}) II }{Jan. 2020 - May. 2020}
      %\resumeItemListStart
      %\resumeItem
      %{Designed and graded assignments for more than 140 students. Held recitations for more than 40 students.}
      
      %\resumeItemListEnd
      \resumeSubHeadingListEnd
      
%-----------PROJECT EXPERIENCE-----------------
\section{Project Experience}
	\resumeSubHeadingListStart
	\resumeSubheading
	{Share Your Review}{Austin, TX}
	{Course project for EE 382V: Advanced Programming Tools at the University of Texas at Austin}{Sept 2019 - Dec 2019}
	\resumeItemListStart
	%\resumeItem
	%{Participated this \textbf{full stack} project which allows users to post their book reviews and share to other apps. Users can create their own account, post book reviews, read reviews from others, subscribe certain categories and get corresponding update notifications. }
	\resumeItem
	{Designed a \textbf{MongoDB} database with \textbf{PyMongo} as our project database. Carried out \textbf{unit testing} for all APIs.
	}
	\resumeItem
	{Developed an \textbf{MVC}-structure backend with \textbf{Flask} \& \textbf{Flask-RESTful}. Used \textbf{Firebase Storage} to store uploaded photos. }
	\resumeItem
	{Accomplished a \textbf{web} frontend with \textbf{HTML/CSS/Bootstrap} and embedded it into our Flask-RESTful backend.}
	\resumeItem
	{Applied \textbf{Kotlin} in our \textbf{Android} frontend. Designed a \textbf{reusable cardview template} as a public method for other team members. Utilized \textbf{Camera API} and \textbf{Location API} to provide our users with diverse uploading choices. Used \textbf{Volley} to handle review creation request.  }
	\resumeItem
	{Implemented the \textbf{React Native} frontend with \textbf{Expo}. Beside of the points mentioned in the Android part, we enabled \textbf{notification} and \textbf{sharing} features based on Expo components. }
	\resumeItemListEnd
	

	  
  \resumeSubHeadingListEnd
  
%%-----------PUBLICATION-----------------
\section{Publications}
Manor, L., \textbf{Su, Y.}, et al. "\textbf{What is FAFSA? Interpreting non-technical jargon in domain-specific text}", COLING 2020, submitted.  
\pubItemListStart
	\resumeItem 
	{Introduced a FASFA \textbf{jargon-definition-context dataset} which enable further research for this common but incomprehensible (especially for outsiders) linguistic phenomenon.} 
	%\resumeItem{Proved a key feature of jargon that jargon phrases are more "supprising" in context by applying \textbf{micro \& macro T-test} for the \textbf{GPT loss distribution} across jargons \& non-jargons.}
	\resumeItem{Proved the \textbf{computational linguistic association} between jargons, jargon definitions and corresponding context based on average \textbf{BERT pretrained embeddings} and made it learnable by using a simple \textbf{bi-linear model}.  }
	
	
\pubItemListEnd
%
%--------PROGRAMMING SKILLS------------
%\section{Programming Skills}
%  \resumeSubHeadingListStart
%    \item{
%      \textbf{Languages}{: Scala, Python, Javascript, C++, SQL, Java}
%      \hfill
%      \textbf{Technologies}{: AWS, Play, React, Kafka, GCE}
%    }
%  \resumeSubHeadingListEnd


%-------------------------------------------
\end{document}
