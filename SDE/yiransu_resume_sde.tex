%-------------------------
% Resume in Latex
% Author : Sourabh Bajaj
% License : MIT
%------------------------

%-------------------------
% To change the vertical indent between items/sections, modify "\vspace" for corresponding objs. 
% To change top indent, modify "\addtolength{\topmargin}{-.65in}"
%-------------------------

\documentclass[letterpaper,11pt]{article}

\usepackage{amssymb}
\usepackage{latexsym}
\usepackage[empty]{fullpage}
\usepackage{titlesec}
\usepackage{marvosym}
\usepackage[usenames,dvipsnames]{color}
\usepackage{verbatim}
\usepackage{enumitem}
\usepackage{makecell}
\usepackage[hidelinks]{hyperref}
\usepackage{fancyhdr}
\usepackage[english]{babel}
\pagestyle{fancy}
\fancyhf{} % clear all header and footer fields
\fancyfoot{}
\renewcommand{\headrulewidth}{0pt}
\renewcommand{\footrulewidth}{0pt}

% Adjust margins
\addtolength{\oddsidemargin}{-0.5in}
\addtolength{\evensidemargin}{-0.5in}
\addtolength{\textwidth}{1in}
\addtolength{\topmargin}{-.8in}
\addtolength{\textheight}{1.5in}

\urlstyle{same}

\raggedbottom
\raggedright
\setlength{\tabcolsep}{0in}

% Sections formatting
\titleformat{\section}{
  \vspace{-10pt}\scshape\raggedright\large
}{}{0em}{}[\color{black}\titlerule \vspace{-5pt}]

%-------------------------
% Custom commands
\newcommand{\resumeItem}[1]{
  \item\small{
    {#1 \vspace{-2pt}}
  }
}


\newcommand{\resumeSubheading}[4]{
  \vspace{-1pt}\item
    \begin{tabular*}{0.97\textwidth}[t]{l@{\extracolsep{\fill}}r}
      \textbf{#1} & #2 \\
      \textit{\small#3} & \textit{\small #4} \\
    \end{tabular*}\vspace{-5pt}
}

\newcommand{\courseworkSubheading}[2]{
	\vspace{-1pt}\item
	\begin{tabular*}{0.97\textwidth}[t]{l@{\extracolsep{\fill}}r}
		\textbf{#1} & #2 \\
	\end{tabular*}\vspace{-5pt}
}

\newcommand{\internSubheading}[2]{
	\vspace{-1pt}\item
	\begin{tabular*}{0.97\textwidth}[t]{l@{\extracolsep{\fill}}r}
		\textbf{#1} & #2 \\
	\end{tabular*}\vspace{-6.5pt}
}

\newcommand{\resumeSubItem}[2]{\resumeItem{#1}{#2}\vspace{-4pt}}

\renewcommand{\labelitemii}{$\circ$}

\newcommand{\resumeSubHeadingListStart}{\begin{itemize}[leftmargin=*]}
\newcommand{\resumeSubHeadingListEnd}{\end{itemize}}
\newcommand{\resumeItemListStart}{\begin{itemize}}
\newcommand{\resumeItemListEnd}{\end{itemize}\vspace{-5pt}}
\newcommand{\pubItemListStart}{\begin{itemize}\setlength{\itemsep}{0pt} \setlength{\parsep}{0pt} \setlength{\parskip}{0pt}}
\newcommand{\pubItemListEnd}{\end{itemize}\vspace{-5pt}}

%-------------------------------------------
%%%%%%  CV STARTS HERE  %%%%%%%%%%%%%%%%%%%%%%%%%%%%


\begin{document}

%----------HEADING-----------------
\begin{tabular*}{\textwidth}{l@{\extracolsep{\fill}}r}
  \textbf{{}{\Large Yiran Su}} & 
  {6805 Wood Hollow Dr}\\
  {\href{mailto:yiransucdr@gmail.com}{\underline{\smash{yiransucdr@gmail.com}} |}
  {\underline{\smash{\url{linkedin.com/in/yiransucdr}}} } } & {Austin, TX} 
\end{tabular*}


%-----------EDUCATION-----------------
\section{Education}
	\resumeSubHeadingListStart
		\resumeSubheading
			{The University of Texas at Austin}{Austin, TX, USA}
			{\makecell[tl]{\textbf{M.S.} in Engineering,  \textbf{Software Engineering and System} major, \textbf{GPA}: 3.76/4 }}
			{Aug. 2019 - May. 2021}
		\resumeSubheading
			{Sun Yat-sen University, School of Data and Computer Science}{Guangzhou, China}
			{\makecell[tl]{\textbf{B.E.} in Network Engineering, top 10\% in the class.}}
			{Aug. 2015 - Jun. 2019}
			\resumeItemListStart
			\resumeItem 
				{\textbf{Course Highlights} Data Structure \& Algorithms, Operating System, Computer Network, Web Programming}
			\resumeItemListEnd
\resumeSubHeadingListEnd

%-----------SKILLS-----------------
\section{Skills}
\textbf{Programming Language~}{C++, Python, Java, SQL, HTML, CSS, JavaScript, Kotlin, Shell, GoLang}\\
\textbf{Framework and Tools~}{React Native, Flask, PyTorch, Tensorflow, Docker, MongoDB, Kubernetes}

%-----------AWARDS AND HONORS-----------------
%\section{Awards And Honors}
%\resumeSubHeadingListStart
%\resumeSubItem{Innovative Design Award in the 2018 International Aerial Robotics Competition (2018 IARC)}{}
%\resumeSubItem{Honorable Award in the 2018 Mathematical Contest in Modeling (2018 MCM)}{}
%\resumeSubItem{The First Prize Scholarship of Sun Yat-sen University(Top 5\%)}{}
%\resumeSubHeadingListEnd

%-----------WORK EXPERIENCE-----------------
\section{Intern Experience}
  \resumeSubHeadingListStart
        
  \internSubheading
  {Graduate Research Assistant(C-PAC)}
  {Dell Medical School, UT-Austin, Jan. 2021 - May. 2021}
  \resumeItemListStart
  \resumeItem
  {C-PAC is a configurable processing pipeline for functional brain MRI data. My work is focusing on the configurable frontend of C-PAC, using \textbf{React.js} (JavaScript) with React-Redux and React-Saga.}
  \resumeItemListEnd
  
  	\internSubheading
  	{Coherent Logix Inc.}{Software Dev Intern, May. 2020 - Aug. 2020, Austin, USA}
  	\resumeItemListStart
  	\resumeItem
  	{Worked on \textbf{neural network quantization}, which simplifies the original neural network model, and persists the original model precision.}
  	\resumeItem
  	{Accomplished \textbf{16/8/4-bit} Quantization Aware Training (\textbf{QAT}) with \textbf{Tensorflow 2} (\textbf{Python}). }
  	\resumeItem
  	{Developed a set of \textbf{QAT generation \& validation APIs} in \textbf{TensorFlow 2} (\textbf{Python}) to simplify future applications. }
  	\resumeItemListEnd
  	
    \internSubheading
      {Tencent Inc.}{Software Dev Intern, Sept. 2018 - Mar. 2019, Shenzhen, China}
      \resumeItemListStart
      	\resumeItem
      	  {Focused on a pattern-based natural language parsing framework in \textbf{Python} for a task-oriented chatbot. }
      	\resumeItem
          {Deployed the parsing framework on a \textbf{Tornado} (Python) Server. The framework was able to handle \textbf{100k+} user vocal requests per day, with a \textbf{27\%} latency drop.} 
        \resumeItem
          {Created an \textbf{automatic} log analyzer in \textbf{Python} to evaluate user behavior (customer stickiness, feature performance, etc.)}
      \resumeItemListEnd
      
      \internSubheading
      {Graduate Teaching Assistant}
      {EE 422C (\textbf{Java}), UT-Austin, Jan. 2020 - May. 2020}
      \resumeItemListStart
      \resumeItem
      {Led weekly lectures on Java, including polymorphism, Java generic, multithreading, lambda, stream, etc. }
      %\resumeItem{Designed and graded assignments \& exams, as well as participated in various teaching negotiations.}
      \resumeItemListEnd
      \resumeSubHeadingListEnd
      
%-----------PROJECT EXPERIENCE-----------------
\section{Project Experience}
	\resumeSubHeadingListStart
	\internSubheading
	{Share Your Review}{Course project for EE 382V at UT-Austin, Sept 2019 - Dec 2019}
	\resumeItemListStart
	%\resumeItem
	%{Participated this \textbf{full stack} project which allows users to post their book reviews and share to other apps. Users can create their own account, post book reviews, read reviews from others, subscribe certain categories and get corresponding update notifications. }
	\resumeItem
	{Worked on a full stack project that requires both frontend (web/Andriod/ReactNative) and backend (response framework /database) implementation. The project goal was to help people to connect with nearby book readers.}
	\resumeItem
	{Designed a \textbf{MVC}-styled \textbf{Flask} \& \textbf{Flask-RESTful} (\textbf{Python}) backend. \textbf{Firebase} and \textbf{PyMongo} for database. }
	\resumeItem
	{Accomplished a \textbf{web} frontend with \textbf{HTML/CSS/Bootstrap} and embedded it into the Flask-RESTful backend.}
	\resumeItem
	{Applied \textbf{Kotlin} for the \textbf{Android} frontend. Designed a \textbf{reusable cardview template} for the team. Utilized \textbf{Volley} to handle requests. Kotlin's \textbf{Camera \& Location APIs} were applied.  }
	\resumeItem
	{Implemented a \textbf{React Native} (JavaScript) frontend, with Camera, location, sharing and notification features. }
	\resumeItemListEnd
	
	\internSubheading{Sun Yat-sen University Club Information Platform}{Mar 2018 - Jun 2018}
	\resumeItemListStart
	\resumeItem{Designed the app to spread club event information across campus. Contributed as a frontend developer.}
	\resumeItem{Constructed the \textbf{Android} frontend in \textbf{Java}, with \textbf{Retrofit2} \& \textbf{RxJava} applied. Designed the Web frontend in \textbf{HTML/CSS/JavaScript}. }		
	\resumeItemListEnd
	\internSubheading{2018 International Aerial Robotics Competition}{Computer Vision, Sept 2017 - Aug 2018}
	\resumeItemListStart
	\resumeItem{For ground object detection, designed an \textbf{OpenCV}-based Support Vector Machine (SVM) classifier, with a self-implemented Histogram of Oriented Gradient (HOG) descriptor in \textbf{C++}. }
	\resumeItemListEnd
	% \internSubheading{Distributed System Design}{MIT 6.824 online open course, Sept 2020 - present}
% 	\resumeItemListStart
%	\resumeItem{Implementing \textbf{MapReduce} and \textbf{Raft} in \textbf{Go} to utilize distributed processing.  }		
%	\resumeItemListEnd	  
  \resumeSubHeadingListEnd
  
%%-----------PUBLICATION-----------------
\section{Publications}
Manor, L., \textbf{Su, Y.}, et al. "\textbf{How to apply for financial aid: Exploring perplexity and jargon in texts for non-expert audiences}", SCiL 2021, accepted.  
\pubItemListStart
	%\resumeItem 
	%{Introduced a FASFA \textbf{jargon-definition-context dataset} which enables further research for this common but incomprehensible (especially for outsiders) linguistic phenomenon.} 
	%\resumeItem{Proved a key feature of jargon that jargon phrases are more "supprising" in context by applying \textbf{micro \& macro T-test} for the \textbf{GPT loss distribution} across jargons \& non-jargons.}
	\vspace{-7pt}\resumeItem{Proved that pre-trained neural language models are less likely to predict jargon phases without fine-tuning, by analyzing the prediction perplexity of \textbf{GPT-2}.}
	\resumeItem{Examined the computational linguistic association between jargons and the corresponding context with \textbf{PyTorch} (\textbf{Python}). BERT embeddings were applied. (This phase was not presented in the paper)}
	
\pubItemListEnd

%
%--------Comments------------
%\vskip 0.075in
%* On ResNet and MobileNet (neural networks) with 16 \& 8-bit QAT, the model accuracy drop was within 3\%.

%** The APIs enable transform from a non-QATed model to a QATed model and compatible with all sequential/functional models(self-defined, loaded, Keras-imported, etc.).

%
%--------PROGRAMMING SKILLS------------
%\section{Programming Skills}
%  \resumeSubHeadingListStart
%    \item{
%      \textbf{Languages}{: Scala, Python, Javascript, C++, SQL, Java}
%      \hfill
%      \textbf{Technologies}{: AWS, Play, React, Kafka, GCE}
%    }
%  \resumeSubHeadingListEnd


%-------------------------------------------
\end{document}
